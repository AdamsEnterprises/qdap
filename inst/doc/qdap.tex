\documentclass[a4paper]{article}\usepackage{graphicx, color}
%% maxwidth is the original width if it is less than linewidth
%% otherwise use linewidth (to make sure the graphics do not exceed the margin)
\makeatletter
\def\maxwidth{ %
  \ifdim\Gin@nat@width>\linewidth
    \linewidth
  \else
    \Gin@nat@width
  \fi
}
\makeatother

\IfFileExists{upquote.sty}{\usepackage{upquote}}{}
\definecolor{fgcolor}{rgb}{0.2, 0.2, 0.2}
\newcommand{\hlnumber}[1]{\textcolor[rgb]{0,0,0}{#1}}%
\newcommand{\hlfunctioncall}[1]{\textcolor[rgb]{0.501960784313725,0,0.329411764705882}{\textbf{#1}}}%
\newcommand{\hlstring}[1]{\textcolor[rgb]{0.6,0.6,1}{#1}}%
\newcommand{\hlkeyword}[1]{\textcolor[rgb]{0,0,0}{\textbf{#1}}}%
\newcommand{\hlargument}[1]{\textcolor[rgb]{0.690196078431373,0.250980392156863,0.0196078431372549}{#1}}%
\newcommand{\hlcomment}[1]{\textcolor[rgb]{0.180392156862745,0.6,0.341176470588235}{#1}}%
\newcommand{\hlroxygencomment}[1]{\textcolor[rgb]{0.43921568627451,0.47843137254902,0.701960784313725}{#1}}%
\newcommand{\hlformalargs}[1]{\textcolor[rgb]{0.690196078431373,0.250980392156863,0.0196078431372549}{#1}}%
\newcommand{\hleqformalargs}[1]{\textcolor[rgb]{0.690196078431373,0.250980392156863,0.0196078431372549}{#1}}%
\newcommand{\hlassignement}[1]{\textcolor[rgb]{0,0,0}{\textbf{#1}}}%
\newcommand{\hlpackage}[1]{\textcolor[rgb]{0.588235294117647,0.709803921568627,0.145098039215686}{#1}}%
\newcommand{\hlslot}[1]{\textit{#1}}%
\newcommand{\hlsymbol}[1]{\textcolor[rgb]{0,0,0}{#1}}%
\newcommand{\hlprompt}[1]{\textcolor[rgb]{0.2,0.2,0.2}{#1}}%

\usepackage{framed}
\makeatletter
\newenvironment{kframe}{%
 \def\at@end@of@kframe{}%
 \ifinner\ifhmode%
  \def\at@end@of@kframe{\end{minipage}}%
  \begin{minipage}{\columnwidth}%
 \fi\fi%
 \def\FrameCommand##1{\hskip\@totalleftmargin \hskip-\fboxsep
 \colorbox{shadecolor}{##1}\hskip-\fboxsep
     % There is no \\@totalrightmargin, so:
     \hskip-\linewidth \hskip-\@totalleftmargin \hskip\columnwidth}%
 \MakeFramed {\advance\hsize-\width
   \@totalleftmargin\z@ \linewidth\hsize
   \@setminipage}}%
 {\par\unskip\endMakeFramed%
 \at@end@of@kframe}
\makeatother

\definecolor{shadecolor}{rgb}{.97, .97, .97}
\definecolor{messagecolor}{rgb}{0, 0, 0}
\definecolor{warningcolor}{rgb}{1, 0, 1}
\definecolor{errorcolor}{rgb}{1, 0, 0}
\newenvironment{knitrout}{}{} % an empty environment to be redefined in TeX

\usepackage{alltt}
\usepackage[american]{babel}
\usepackage{csquotes}
\usepackage{scrextend}
\usepackage{amsmath}
\usepackage[sc]{mathpazo}
\usepackage{geometry}
\geometry{verbose,tmargin=2.5cm,bmargin=2.5cm,lmargin=2.5cm,rmargin=2.5cm}
\setcounter{secnumdepth}{2}
\setcounter{tocdepth}{2}
\usepackage{url}
\usepackage[unicode=true,pdfusetitle,
 bookmarks=true,bookmarksnumbered=true,bookmarksopen=true,bookmarksopenlevel=2,
 breaklinks=false,pdfborder={0 0 1},backref=false,colorlinks=false]
 {hyperref}

\usepackage{hyperref}
\usepackage{here}
\usepackage{tcolorbox}
\usepackage{ulem}


\usepackage{listings}
\usepackage{inconsolata}


\newcommand{\acronym}[1]{\textsc{#1}}
\newcommand{\class}[1]{\mbox{\textsf{#1}}}
\newcommand{\code}[1]{\mbox{\texttt{#1}}}
\newcommand{\pkg}[1]{{\normalfont\fontseries{b}\selectfont #1}}
\newcommand{\proglang}[1]{\textsf{#1}}

\begin{document}

\title{Introduction to \pkg{qdap}:\\Quantitative Discourse Analysis Package}
\author{Tyler W. Rinker}
\maketitle




\section*{Introduction}
\hspace{.4cm} This vignette gives an introduction to basic workflow and function 
usage for \pkg{qdap}. \pkg{qdap} is an \proglang{R} package designed
to assist in quantitative discourse analysis. The package stands as a bridge
between qualitative transcripts of dialogue and statistical analysis and
visualization.
\vspace{.5cm}

\pkg{qdap} automates many of the tasks associated with quantitative 
discourse analysis of transcripts containing discourse including frequency 
counts of sentence types, words, sentence, turns of talk, syllable counts and 
other assorted analysis tasks. The package provides parsing tools for preparing 
transcript data. Many functions enable the user to aggregate data by any number 
of grouping variables providing analysis and seamless integration with other 
\proglang{R} packages that undertake higher level analysis and visualization of 
text. This provides the user with a more efficient and targeted analysis.

\section*{Work Flow}
\begin{enumerate}
  \item Transcribing and Initial Coding
  \item Reading In Transcripts
  \item Viewing
  \item Low Level qdap Tools
  \item Cleaning and Parsing
  \item Shaping   %key_merge sentSplit
  \item Coding (range and time stamp) %output, coding, input, reshaping  all under transcript range and time (ment 
  \item Code Transformations
  \item Quantifying (counts, indices and scores)%  \item Counts  \item Text Scores
  \item Visualizations   %ggplot2, base, grid, lattice
  \item Export Tables, Plots and Data  %save after long parse
  \item Linking to Other Packages  %mention reshape2, plyr, lmer/SEM/lavan/stats/base/
\end{enumerate}

\section*{Transcribing and Coding}
\section*{Reading in Transcripts}

\subsection*{Read a Single Transcript}
\begin{knitrout}
\definecolor{shadecolor}{rgb}{0.969, 0.969, 0.969}\color{fgcolor}\begin{kframe}
\begin{alltt}
doc1 <- \hlfunctioncall{system.file}(\hlstring{"extdata/trans1.docx"}, package = \hlstring{"qdap"})
dat1 <- \hlfunctioncall{read.transcript}(doc1)
\hlfunctioncall{truncdf}(dat1, 40)
\end{alltt}
\end{kframe}
\end{knitrout}


\begin{knitrout}
\definecolor{shadecolor}{rgb}{0.969, 0.969, 0.969}\color{fgcolor}\begin{kframe}
\begin{verbatim}
##                  X1                                       X2
## 1      Researcher 2                         October 7, 1892.
## 2         Teacher 4 Students it's time to learn. [Student di
## 3 Multiple Students        Yes teacher we're ready to learn.
## 4     [Cross Talk 3                                      00]
## 5         Teacher 4 Let's read this terrific book together.
\end{verbatim}
\end{kframe}
\end{knitrout}


\begin{knitrout}
\definecolor{shadecolor}{rgb}{0.969, 0.969, 0.969}\color{fgcolor}\begin{kframe}
\begin{alltt}
dat2 <- \hlfunctioncall{read.transcript}(doc1, col.names = \hlfunctioncall{c}(\hlstring{"person"}, \hlstring{"dialogue"}))
\hlfunctioncall{truncdf}(dat2, 40)
\end{alltt}
\end{kframe}
\end{knitrout}


\begin{knitrout}
\definecolor{shadecolor}{rgb}{0.969, 0.969, 0.969}\color{fgcolor}\begin{kframe}
\begin{verbatim}
##              person                                 dialogue
## 1      Researcher 2                         October 7, 1892.
## 2         Teacher 4 Students it's time to learn. [Student di
## 3 Multiple Students        Yes teacher we're ready to learn.
## 4     [Cross Talk 3                                      00]
## 5         Teacher 4 Let's read this terrific book together.
\end{verbatim}
\end{kframe}
\end{knitrout}


\begin{knitrout}
\definecolor{shadecolor}{rgb}{0.969, 0.969, 0.969}\color{fgcolor}\begin{kframe}
\begin{alltt}
doc3 <- \hlfunctioncall{system.file}(\hlstring{"extdata/trans4.xlsx"}, package = \hlstring{"qdap"})
dat3 <- \hlfunctioncall{read.transcript}(doc3)
\hlfunctioncall{truncdf}(dat3, 40)
\end{alltt}
\end{kframe}
\end{knitrout}


\begin{knitrout}
\definecolor{shadecolor}{rgb}{0.969, 0.969, 0.969}\color{fgcolor}\begin{kframe}
\begin{verbatim}
##                   V1                                       V2
## 1      Researcher 2:                         October 7, 1892.
## 2         Teacher 4:             Students it's time to learn.
## 3               <NA>                                     <NA>
## 4 Multiple Students:        Yes teacher we're ready to learn.
## 5               <NA>                                     <NA>
## 6         Teacher 4: Let's read this terrific book together.
\end{verbatim}
\end{kframe}
\end{knitrout}


\subsection*{Reading Many Transcripts}

\begin{knitrout}
\definecolor{shadecolor}{rgb}{0.969, 0.969, 0.969}\color{fgcolor}\begin{kframe}
\begin{alltt}
DIR <- \hlfunctioncall{gsub}(\hlstring{"trans1.docx"}, \hlstring{""}, \hlfunctioncall{system.file}(\hlstring{"extdata/trans1.docx"}, package = \hlstring{"qdap"}))
\hlfunctioncall{dir_map}(DIR, col.names = \hlfunctioncall{c}(\hlstring{"per"}, \hlstring{"text"}))
\end{alltt}
\end{kframe}
\end{knitrout}






\begin{knitrout}
\definecolor{shadecolor}{rgb}{0.969, 0.969, 0.969}\color{fgcolor}\begin{kframe}
\begin{lstlisting}[basicstyle=\ttfamily,breaklines=true]
## dat1 <- read.transcript('C:/R/R-2.15.1/library/qdap/extdata//trans1.docx', col.names = c('per', 'text'), skip = 0)
## dat2 <- read.transcript('C:/R/R-2.15.1/library/qdap/extdata//trans2.docx', col.names = c('per', 'text'), skip = 0)
## dat3 <- read.transcript('C:/R/R-2.15.1/library/qdap/extdata//trans3.docx', col.names = c('per', 'text'), skip = 0)
## dat4 <- read.transcript('C:/R/R-2.15.1/library/qdap/extdata//trans4.xlsx', col.names = c('per', 'text'), skip = 0)
\end{lstlisting}
\end{kframe}
\end{knitrout}



\section*{Cleaning and Parsing}

\bibliographystyle{abbrvnat}
\bibliography{refs}

\end{document}
